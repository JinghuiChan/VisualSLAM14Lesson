%% Generated by Sphinx.
\def\sphinxdocclass{report}
\documentclass[letterpaper,10pt,english]{sphinxmanual}
\ifdefined\pdfpxdimen
   \let\sphinxpxdimen\pdfpxdimen\else\newdimen\sphinxpxdimen
\fi \sphinxpxdimen=.75bp\relax
\ifdefined\pdfimageresolution
    \pdfimageresolution= \numexpr \dimexpr1in\relax/\sphinxpxdimen\relax
\fi
%% let collapsible pdf bookmarks panel have high depth per default
\PassOptionsToPackage{bookmarksdepth=5}{hyperref}
%% turn off hyperref patch of \index as sphinx.xdy xindy module takes care of
%% suitable \hyperpage mark-up, working around hyperref-xindy incompatibility
\PassOptionsToPackage{hyperindex=false}{hyperref}
%% memoir class requires extra handling
\makeatletter\@ifclassloaded{memoir}
{\ifdefined\memhyperindexfalse\memhyperindexfalse\fi}{}\makeatother

\PassOptionsToPackage{warn}{textcomp}

\catcode`^^^^00a0\active\protected\def^^^^00a0{\leavevmode\nobreak\ }
\usepackage{cmap}
\usepackage{xeCJK}
\usepackage{amsmath,amssymb,amstext}
\usepackage{babel}



\setmainfont{FreeSerif}[
  Extension      = .otf,
  UprightFont    = *,
  ItalicFont     = *Italic,
  BoldFont       = *Bold,
  BoldItalicFont = *BoldItalic
]
\setsansfont{FreeSans}[
  Extension      = .otf,
  UprightFont    = *,
  ItalicFont     = *Oblique,
  BoldFont       = *Bold,
  BoldItalicFont = *BoldOblique,
]
\setmonofont{FreeMono}[
  Extension      = .otf,
  UprightFont    = *,
  ItalicFont     = *Oblique,
  BoldFont       = *Bold,
  BoldItalicFont = *BoldOblique,
]



\usepackage[Sonny]{fncychap}
\ChNameVar{\Large\normalfont\sffamily}
\ChTitleVar{\Large\normalfont\sffamily}
\usepackage{sphinx}

\fvset{fontsize=\small,formatcom=\xeCJKVerbAddon}
\usepackage{geometry}


% Include hyperref last.
\usepackage{hyperref}
% Fix anchor placement for figures with captions.
\usepackage{hypcap}% it must be loaded after hyperref.
% Set up styles of URL: it should be placed after hyperref.
\urlstyle{same}

\addto\captionsenglish{\renewcommand{\contentsname}{Contents:}}

\usepackage{sphinxmessages}
\setcounter{tocdepth}{1}



\title{VisualSLAM14Lesson}
\date{2022 年 02 月 14 日}
\release{1.0}
\author{jinghui chan}
\newcommand{\sphinxlogo}{\vbox{}}
\renewcommand{\releasename}{发布}
\makeindex
\begin{document}

\ifdefined\shorthandoff
  \ifnum\catcode`\=\string=\active\shorthandoff{=}\fi
  \ifnum\catcode`\"=\active\shorthandoff{"}\fi
\fi

\pagestyle{empty}
\sphinxmaketitle
\pagestyle{plain}
\sphinxtableofcontents
\pagestyle{normal}
\phantomsection\label{\detokenize{index::doc}}



\chapter{Prerequisites}
\label{\detokenize{Prerequisites/prerequisites:prerequisites}}\label{\detokenize{Prerequisites/prerequisites::doc}}

\section{安装 Ubuntu 操作系统}
\label{\detokenize{Prerequisites/InstallOS:ubuntu}}\label{\detokenize{Prerequisites/InstallOS::doc}}

\section{Sphinx+Github+ReadtheDoc 的使用}
\label{\detokenize{Prerequisites/Sphinx+Github+ReadtheDoc:sphinx-github-readthedoc}}\label{\detokenize{Prerequisites/Sphinx+Github+ReadtheDoc::doc}}

\section{安装 OpenCV}
\label{\detokenize{Prerequisites/InstallOpenCV:opencv}}\label{\detokenize{Prerequisites/InstallOpenCV::doc}}

\section{安装 Pangolin}
\label{\detokenize{Prerequisites/InstallPangolin:pangolin}}\label{\detokenize{Prerequisites/InstallPangolin::doc}}

\section{其他}
\label{\detokenize{Prerequisites/Others:id1}}\label{\detokenize{Prerequisites/Others::doc}}

\chapter{SLAM后端学习}
\label{\detokenize{ch9/ch9:slam}}\label{\detokenize{ch9/ch9::doc}}

\section{ch9学习计划}
\label{\detokenize{ch9/ch9_u5b66_u4e60_u8ba1_u5212:ch9}}\label{\detokenize{ch9/ch9_u5b66_u4e60_u8ba1_u5212::doc}}
\begin{sphinxadmonition}{important}{重要:}
\sphinxAtStartPar
本章节学习视觉SLAM后端相关知识,要求了解状态估计相关内容,并熟练掌握g2o、Ceres第三方库求解BA。此章节要求动手能力较强!!!
\end{sphinxadmonition}


\subsection{知识回顾}
\label{\detokenize{ch9/ch9_u5b66_u4e60_u8ba1_u5212:id1}}
\sphinxAtStartPar
回顾一下内容:
\begin{itemize}
\item {} 
\sphinxAtStartPar
BA相关知识

\item {} 
\sphinxAtStartPar
g2o库的使用方法

\item {} 
\sphinxAtStartPar
Ceres库的使用方法

\end{itemize}


\subsection{学习内容}
\label{\detokenize{ch9/ch9_u5b66_u4e60_u8ba1_u5212:id2}}\begin{itemize}
\item {} 
\sphinxAtStartPar
状态估计

\item {} 
\sphinxAtStartPar
BA与图优化

\item {} 
\sphinxAtStartPar
实践:g2o求解BA

\item {} 
\sphinxAtStartPar
实践:Ceres求解BA

\end{itemize}


\subsection{具体安排}
\label{\detokenize{ch9/ch9_u5b66_u4e60_u8ba1_u5212:id3}}

\subsubsection{1. 状态估计}
\label{\detokenize{ch9/ch9_u5b66_u4e60_u8ba1_u5212:id4}}
\sphinxAtStartPar
时间安排: 2022年2月14日 – 2022年2月18日

\sphinxAtStartPar
重难点:
\begin{itemize}
\item {} 
\sphinxAtStartPar
理解状态估计原理

\end{itemize}


\subsubsection{2. BA与图优化}
\label{\detokenize{ch9/ch9_u5b66_u4e60_u8ba1_u5212:ba}}
\sphinxAtStartPar
时间安排: 2022年2月14日 – 2022年2月18日

\sphinxAtStartPar
重难点:
\begin{itemize}
\item {} 
\sphinxAtStartPar
投影模型和BA代价函数

\item {} 
\sphinxAtStartPar
BA的求解

\end{itemize}


\subsubsection{3. 实践:Ceres求解BA}
\label{\detokenize{ch9/ch9_u5b66_u4e60_u8ba1_u5212:ceresba}}
\sphinxAtStartPar
时间安排: 2022年2月21日 – 2022年2月25日

\sphinxAtStartPar
重难点:
\begin{itemize}
\item {} 
\sphinxAtStartPar
Ceres库的使用

\item {} 
\sphinxAtStartPar
编程实现使用Ceres库求解BA的Demo

\end{itemize}


\subsubsection{4. 实践:g2o求解BA}
\label{\detokenize{ch9/ch9_u5b66_u4e60_u8ba1_u5212:g2oba}}
\sphinxAtStartPar
时间安排: 2022年2月21日 – 2022年2月25日

\sphinxAtStartPar
重难点:
\begin{itemize}
\item {} 
\sphinxAtStartPar
g2o库的使用

\item {} 
\sphinxAtStartPar
编程实现使用g2o库求解BA的Demo

\end{itemize}


\subsubsection{5. 文档}
\label{\detokenize{ch9/ch9_u5b66_u4e60_u8ba1_u5212:id5}}
\sphinxAtStartPar
学习过程中随手记录文档,包括但不局限于:
\begin{itemize}
\item {} 
\sphinxAtStartPar
学习中遇到的问题、困惑

\item {} 
\sphinxAtStartPar
代码注解

\item {} 
\sphinxAtStartPar
公式推导

\item {} 
\sphinxAtStartPar
重点语句的理解

\item {} 
\sphinxAtStartPar
感悟

\item {} 
\sphinxAtStartPar
对组员或团队的建议

\end{itemize}


\section{状态估计}
\label{\detokenize{ch9/_u72b6_u6001_u4f30_u8ba1:id1}}\label{\detokenize{ch9/_u72b6_u6001_u4f30_u8ba1::doc}}

\section{BA与图优化}
\label{\detokenize{ch9/BA_u4e0e_u56fe_u4f18_u5316:ba}}\label{\detokenize{ch9/BA_u4e0e_u56fe_u4f18_u5316::doc}}

\section{实践:Ceres解BA}
\label{\detokenize{ch9/_u5b9e_u8df5:Ceres_u89e3BA:ceresba}}\label{\detokenize{ch9/_u5b9e_u8df5:Ceres_u89e3BA::doc}}

\section{实践:g2o求解BA}
\label{\detokenize{ch9/_u5b9e_u8df5:g2o_u6c42_u89e3BA:g2oba}}\label{\detokenize{ch9/_u5b9e_u8df5:g2o_u6c42_u89e3BA::doc}}


\renewcommand{\indexname}{索引}
\printindex
\end{document}